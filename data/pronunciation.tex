\documentclass[11pt]{amsart}
\usepackage{geometry}
\geometry{letterpaper}
\usepackage{graphicx}
\usepackage{amssymb}
\usepackage{epstopdf}
\usepackage{tipa}
\DeclareGraphicsRule{.tif}{png}{.png}{`convert #1 `dirname #1`/`basename #1 .tif`.png}

\begin{document}

\thispagestyle{empty}

\begin{tabular}{cccc}
Letter & Sound in IPA & Example Word & Explanation \\
A a & /a/ & ami & the "a" in "father" \\
B b & /b/ & birdo & the "b" in "bed" \\
C c & /\t{ts}/ & cent & like the "ts" in "cats"* \\
\^C \^c & /\textipa{\t{tS}}/ & \^cambro & the "ch" in "church" \\
D d & /d/ & domo & the "d" in "dog" \\
E e & /e/ & en & the "e" in "let" \\
F f & /f/ & fi\^so & the "f" in "fin" \\
G g & /g/ & granda & the "g" in "gone" (NOT the "g" in "gin") \\
\^G \^g & /\textipa{\t{dZ}}/ & \^gi & the "g" in "general" (NOT the "g" in "go") \\
H h & /h/ & hundo & the "h" in "house" \\
\^H \^h & /x/ & \^horo & like the "ch" in German "ach"* \\
I i & /i/ & ili & the "ee" in "feel" \\
J j & /j/ & jes & the "y" in "yes" \\
\^J \^j & /\textipa{Z}/ & \^ja\u{u}do & the "s" in "pleasure" \\
K k & /k/ & kato & the "k" in "king" \\
L l & /l/ & la & the "l" in "long" \\
M m & /m/ & muso & the "m" in "mother" \\
N n & /n/ & ni & the "n" in "not" \\
O o & /o/ & ovo & the "o" in "for" \\
P p & /p/ & persono & the "p" in "pare" \\
R r & /r/ & ri & like "rr" in Spanish or Italian* \\
S s & /s/ & saluton & the "s" in "sing" (NOT the "s" in "rise") \\
\^S \^s & /S/ & \^sipo & the "sh" in "she" \\
T t & /t/ & tapi\^so & the "t" in "top" (NOT the "tt" in "latter") \\
U u & /u/ & uzas & the "u" in "rude" \\
\u{U} \u{u} & /w/ & na\u{u} & the final "w" sound heard in "ow!" \\
V v & /v/ & voli & the "v" in "vine" \\
Z z & /z/ & zipo & the "z" in "zip" \\
\end{tabular}

~\

*these sounds don't occur in English. However, they aren't that hard to master; also, \^h is very rare in Esperanto, and you probably won't have to worry about it for a while. It's possible to speak the language without dealing with this phoneme. However, r and c are all over the place. 

~\

Finally, the stress in an Esperanto word always comes on the second-to-last syllable. This rule is always consistant, without fail. The letters never change their pronunciations --- you don't have to worry about "I before E except after C" or anything of the sort. 

\end{document}
\bye
